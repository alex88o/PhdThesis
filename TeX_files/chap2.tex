\chapter{Crowdsourced Media Analysis}\label{ch2}
With the rapid growth in communication technology, both companies and research institutes have been given the opportunity to perform large scale analysis on a multitude of real user-generated data, with a huge variety of application contexts.
Crowdsourcing provides the opportunity for input from a number of sources, with different degrees of granularity. It allows organizations to develop solutions for both strategic issues and a method to find new ways to reach audiences on a broader scale. Moreover, the growing industry of online communication through smartphones provides a way for people of all backgrounds to give input on a project or research.
%Sometimes the possibility to participate on a project, or just express their opinions encourages people to partecipate to crowdsourcing campains.
%Often times, the trigger to attract a huge amount of contributors is the so called ``gamification''.
Social media, blogs, forums, comment sections in online websites allow the opportunity for people to give suggestions or concerns.

There are three main assets that supported the rise of the ``crowdsourcing era'':
\begin{itemize}
	\item \textbf{Social Platforms}: the diffusion of social networks plays a crucial role in collecting information about people opinion, trends and behaviour. There are general social networks like Facebook where people chat, read news, and share their experiences. Furthermore, there are also very specific social platforms aimed to bring together people with common interests. There are platforms by which computer engineers share code and advices, or professional photographers can share their photos, etc. What happens now is that people love sharing their information, tell friends what they are doing and how they feel. And what is very important for the scientific community is that most of these information are public and immediately available.
	\item \textbf{High Bandwidth Connection}: the number of people with an Internet connection is increasing, as well as the bandwidth and the available connection speed. With the 5G connection, it's possible to download an high quality two hour long movie in less than 4 seconds. 
	The connectivity improvements allowed the development of new services based on the transmission of huge amount of data, and real-time services. This allowed, for instance, web-based services like Netflix and the IP television, with the possibility to watch movies or live events with very high quality and low latency, or to perform a video of the event the user is attending allowing him to share the live streaming through a social network.
	\item \textbf{Personal Devices}: the diffusion of personal devices like smartphones allows people to be connected in every second of their lives, wherever they are in that moment. This allows the users to access on-line services in any moment of their daytime. Moreover, the amount of personal data that can be acquired by personal devices allow these services to be more pervasive and user centric.% Moreover, with the huge amount of available personal data, the provided services became more pervasive and user specific (i.e., more persuasive, and attractive). 
\end{itemize}

%Companies have caught on to the attention surrounding crowdsourcing, and have sought out innovative 
Companies have been attempting innovative ways to get their customers involved both in production and promotion processes of their products and services. Crowdsourcing brings people together through a web-based platform, %this allows brands access to untapped talent that might not be located in their area. Crowdsourcing 
generally by means of social media, so businesses can obtain insights about what topics consumers are talking about or are interested in.
%From asking followers their preference on a design or color scheme, all the way to submitting their own content, companies large and small have realized the benefits of using their most valuable asset (their customers) to test new ideas. 
Asking what people like before offering a new product on the market helps reduce the risk of a product or service failure, while also generating hype around a new offering.

In the last decade, several companies exploited the crowdsourcing paradigm to offer innovative services. For example, crowdsourcing has changed the way people travel. The rise of services like AirBnB, Uber, and what has been termed the ``sharing economy'', transformed what had been primarily a mass-produced experience into a peer-to-peer economic network.

Companies like AirBnB and Uber have driven down prices by increasing the marketplace offer. Customers also benefit from increased variety and personalization in their travel options. 
%Whereas in the past a traveller relied on known hotel and taxi brands to provide reliable experiences, nowadays they are just as likely to tap into the marketplace that one of those new companies facilitates (Uber, AirBnB), and enjoy a more unique experience. 
The traveller’s issues and habits has remained rather the same, what have changed are only the service providers, and often times the service provided. %Rather than the traditional taxi to the airport, flight to a destination and taxi to the hotel pattern, a traveler these days is more likely to take an Uber (or Lyft) to the airport, fly to their destination, hop into another crowd-sourced vehicle, and head to the apartment they booked on AirBnB.
%Additionally, whereas in the past someone new to a city might rely on their hotel’s receptionist for restaurant recommendations, these days they are just as likely to ask Yelp, or even their AirBnB hosts.
Although the low prices can be attractive, the most of users trust the deals of such kind of companies due to the feedbacks of previous customers. Indeed, they do not actually trust the companies, but the opinions of other users of the community (preferably a large amount of them, specially if they are expert users of the platform who provided useful and fair feedbacks). On the other hand, these companies push users to public comments, express their opinions and tell their experiences by exploiting the ``gamification'' approach: the more you contribute, the more you earn (in terms of discounts, reputation, platform tools). 

Besides new emerging companies, also the main IT companies have sought out innovative ideas to exploit crowdsourcing. Google exploits its users' contributions to improve the quality of Google Translate results, and the GPS locations transmitted by a large number of users' smartphones to infer traffic conditions in real time on major roads and highways. 
In 2008, Facebook has exploited crowdsourcing to create different language versions of its website~\cite{dolmaya2011ethics}. %The company claimed this method offers the advantage of providing site versions that are more compatible with local cultures.

The amount of public available and large-scale information supports the study and development of systems able to translate crowdsourced data into clear actionable insights.
\\PARLARE DI IMMAGINI E VIDEO INTRODUCENDO I LAVORI SUCCESSIVI